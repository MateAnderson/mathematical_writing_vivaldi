\documentclass[11pt]{article}
\usepackage{amssymb}
\usepackage{textcomp}
\newcommand{\numpy}{{\tt numpy}}    % tt font for numpy

\topmargin -1in
\textheight 9in
\oddsidemargin -.25in
\evensidemargin -.25in
\textwidth 7in

\begin{document}

\author{Mohammadreza Ardestani 9513004}
\title{Special languages module, Polytechnic university of Tehran}
\maketitle

\medskip
	{\textbf{\huge Exercise 1:}}
% ========== Begin answering questions here
\begin{enumerate}
\item
There are 3 special cases. \textrightarrow \hspace*{0.25cm}  There are three special cases.
\item
X is a finite set. \textrightarrow \hspace*{0.25cm} The set X is finite.
\item
It does not tend to infinaty. \textrightarrow \hspace*{0.25cm} It does not tend to infinity.
\item
It follows $x-1=y^{4}$. \textrightarrow \hspace*{0.25cm} hence(it follows that) $x-1=y^{4}$.
\item
$\therefore c^{-1}$ is undefined. \textrightarrow \hspace*{0.25cm} therefore $c^{-1}$ is undefined.
\item
The product of 2 negatives is positive. \textrightarrow \hspace*{0.25cm} Multiplying two negative numbers results in a positive number. (The product of two negative numbers is a positive number.)
\item
We square the equation. \textrightarrow \hspace*{0.25cm} We square both sides of the equation.
\item
We have less solutions than we had before. \textrightarrow \hspace*{0.25cm}	We have fewer solutions than we had before.
\item
$x^{2} = y^{2}$ are two othogonal lines. \textrightarrow \hspace*{0.25cm} The equation $x^{2}=y^{2}$ represents two orthogonal lines.
\item
Let us device a strategy for a proof. \textrightarrow \hspace*{0.25cm} here is a strategy for the proof. (Let us consider/use/find a strategy for the proof.)
\item
This set of matrixs are all invertible. \textrightarrow \hspace*{0.25cm} This set consists of invertible matrices.
\item
If the integral = 0 the function is undefined. \textrightarrow \hspace*{0.25cm} If the integral equals 0, then the function is undefined.
\item
Purely imaginary is when the real part is zero. \textrightarrow \hspace*{0.25cm} Purely imaginary means that the real part is zero. (If the real part is zero, then the number is purely imaginary.)
\item
Construct the set of vertex of triangles. \textrightarrow \hspace*{0.25cm}  Construct the set of all vertexes of triangles.
\item
From the fact that x = 0, I can't divide by x. \textrightarrow \hspace*{0.25cm} Since x is zero, I can’t divide by x. (I can't divide by $x$, because $x=0$.)
\item
A circle is when major and minor axis are the same. \textrightarrow \hspace*{0.25cm} A circle is a shape with equal major and minor axes.
\item
The function f is not discontinuous. \textrightarrow \hspace*{0.25cm} The function f is continuous.
\item
Plug-in that expression in the other equation. \textrightarrow \hspace*{0.25cm} Add that expression to both sides of the other equation.
\item
I found less solutions than I expected. \textrightarrow \hspace*{0.25cm} I found fewer solutions than I expected.
\item
When the discriminant is $<$ 0, you get complex. \textrightarrow \hspace*{0.25cm} You get complex results if the discriminant is negative. (If the discriminant is negative, you get complex results.)
\item
We prove Euler theorom. \textrightarrow \hspace*{0.25cm} We prove Euler's theorem.
\item
The definate integral is where you don't have integration limits. \textrightarrow \hspace*{0.25cm} An integral is definite when it doesn't have integration limits.
(If an integral doesn't have integration limits, it is definite.)
\item
The asyntotes of this hiperbola are othogonal. \textrightarrow \hspace*{0.25cm} The asymptotes of this hyperbola are orthogonal.
\item
A quadratic function has 1 stationery point. \textrightarrow \hspace*{0.25cm} A quadratic function has only one stationary point.
\item
The solution is not independent of s. \textrightarrow \hspace*{0.25cm} The solution depends on s.
\item
a is negative $\therefore\sqrt{a}$ is complex. \textrightarrow \hspace*{0.25cm} a is negative, therefore $\sqrt{a}$(square root of a) is complex.
\item
Thus x = a. (We assume that a is positive). \textrightarrow \hspace*{0.25cm} Thus x = a. (We assume that a is positive.)
\item
Each value is greater than their reciprical. \textrightarrow \hspace*{0.25cm} Each value is greater than its reciprocal.
\item
Remember to always check the sign. \textrightarrow \hspace*{0.25cm} Always remember to check the sign.
\item
Differentiate f n times. \textrightarrow \hspace*{0.25cm}  	Differentiate function f, repeating this process n times.
\end{enumerate}
\end{document}
